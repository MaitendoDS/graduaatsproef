%---------- Inleiding ---------------------------------------------------------

% TODO: Is dit voorstel gebaseerd op een paper van Research Methods die je
% vorig jaar hebt ingediend? Heb je daarbij eventueel samengewerkt met een
% andere student?
% Zo ja, haal dan de tekst hieronder uit commentaar en pas aan.

%\paragraph{Opmerking}

% Dit voorstel is gebaseerd op het onderzoeksvoorstel dat werd geschreven in het
% kader van het vak Research Methods dat ik (vorig/dit) academiejaar heb
% uitgewerkt (met medesturent VOORNAAM NAAM als mede-auteur).
% 

\section{Inleiding}%
\label{sec:inleiding}

Mijn graduaatsproef richt zich op de ontwikkeling van een mobiele applicatie die gebruikers ondersteunt bij het bijhouden van menstruatiecycli, eet- en drinkgewoonten, pijnervaringen en medicatie-inname. Dit project komt voort uit een persoonlijke behoefte aan een geïntegreerd platform waarin al deze gegevens eenvoudig en overzichtelijk kunnen worden opgeslagen. De doelgroep bestaat uit personen met chronische aandoeningen of hormonale schommelingen die medische gegevens moeten bijhouden, evenals gezondheidsbewuste gebruikers die hun levensstijl willen monitoren.

De centrale probleemstelling luidt: "Hoe kan een mobiele applicatie bijdragen aan een efficiënte en overzichtelijke registratie van menstruatiecycli, voeding, pijnklachten en medicatie-inname?" Momenteel moeten gebruikers vaak meerdere applicaties of fysieke notities gebruiken, wat leidt tot een gefragmenteerd overzicht en mogelijke dataverlies. De onderzoeksdoelstelling is daarom het ontwikkelen van een gebruiksvriendelijke applicatie die deze gegevens op een logische en samenhangende manier samenbrengt. Dit onderzoek zal resulteren in een functionele mobiele applicatie die de registratie en visualisatie van deze gegevens optimaliseert.

%---------- Stand van zaken ---------------------------------------------------

\section{Literatuurstudie}%
\label{sec:literatuurstudie}
Het bijhouden van gezondheidsgegevens is een essentieel onderdeel van zelfzorg en medische opvolging. Menstruatietracking-apps kunnen helpen bij het identificeren van patronen in de cyclus en hormonale schommelingen. Voedingsdagboeken kunnen bijdragen aan een gezondere levensstijl en betere medische diagnoses. Pijnregistratie wordt vaak toegepast bij chronische aandoeningen, waarbij een gedetailleerde logboekfunctie nuttig kan zijn voor medische professionals. Ten slotte zijn medicatieherinneringen zeer handig voor zowel anticonceptie als andere belangrijke medicatie. 

Hoewel er al verschillende apps bestaan die zich focussen op afzonderlijke aspecten van gezondheidstracking, ontbreekt een geïntegreerde oplossing die al deze functies combineert. Dit onderzoek richt zich op de ontwikkeling van een applicatie die deze gegevens op een gecentraliseerde manier beheert, waardoor gebruikers een beter inzicht krijgen in hun gezondheid.

%---------- Methodologie ------------------------------------------------------
\section{Methodologie}%
\label{sec:methodologie}
Ik ben van plan om Flutter te gebruiken om mijn mobiele applicatie te maken.
%---------- Verwachte resultaten ----------------------------------------------
\section{Verwacht resultaat, conclusie}%
\label{sec:verwachte_resultaten}

Het verwachte resultaat van deze graduaatsproef is een functionele mobiele applicatie die menstruatiecycli, voeding, pijnervaringen en medicatie-inname overzichtelijk bijhoudt. Door deze geïntegreerde aanpak krijgen gebruikers een volledig beeld van hun gezondheid en kunnen ze beter geïnformeerde beslissingen nemen.

De meerwaarde van dit onderzoek ligt in de verbetering van zelfmonitoring en medische opvolging. Gebruikers zullen in staat zijn om eenvoudig en gestructureerd gegevens te verzamelen, wat hen helpt bij gesprekken met zorgverleners en het identificeren van gezondheidsproblemen. Daarnaast draagt deze applicatie bij aan de digitalisering en automatisering van gezondheidsregistratie, wat het gebruiksgemak en de efficiëntie verhoogt.