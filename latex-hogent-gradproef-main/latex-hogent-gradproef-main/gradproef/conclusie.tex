%%=============================================================================
%% Conclusie
%%=============================================================================

\chapter{Conclusie}%
\label{ch:conclusie}

Het doel van deze graduaatsproef was het ontwikkelen van een mobiele applicatie die gebruikers met chronische gezondheidsproblemen ondersteunt bij het registreren van menstruatiecycli, voeding, pijnklachten en medicatie-inname. Deze onderzoeksvraag werd beantwoord door middel van een combinatie van literatuuronderzoek, analyse van bestaande applicaties en het ontwikkelen van een werkend prototype met Flutter.

Uit het onderzoek bleek dat er een duidelijke nood bestaat aan centralisatie van gezondheidstracking. Bestaande apps zijn vaak gefragmenteerd en beperken zich tot één specifiek domein, wat het voor gebruikers moeilijk maakt om verbanden te leggen tussen verschillende gezondheidsfactoren. De Unda Health Tracker biedt hierop een antwoord door meerdere trackers in één overzichtelijke app te combineren.

De keuze voor Flutter als ontwikkeltool heeft zich in de praktijk bewezen als een geschikte keuze. Dankzij de declaratieve structuur, het gebruik van herbruikbare widgets en de mogelijkheid om performant te compileren naar native code, kon een applicatie gebouwd worden die intuïtief aanvoelt en op meerdere platformen draait.

De ontwikkelde app maakt het mogelijk voor gebruikers om snel en eenvoudig hun gezondheidsgegevens in te voeren, te visualiseren en eventueel te delen met zorgverleners. Deze functionaliteit draagt bij aan een beter zelfmanagement van chronische klachten, en biedt ook voor artsen een meer gestructureerde en objectieve kijk op subjectieve symptomen zoals pijn of vermoeidheid.

Hoewel het prototype een solide basis vormt, zijn er nog verschillende aspecten die verder kunnen worden uitgewerkt. Zo kan de app uitgebreid worden met gepersonaliseerde meldingen, synchronisatie met wearables, en integratie van AI om automatisch patronen te herkennen. Verder onderzoek zou zich ook kunnen richten op gebruikersonderzoek met grotere testgroepen en langdurige evaluatie in een klinische context.

Samengevat toont deze graduaatsproef aan dat een centrale, gebruiksvriendelijke mobiele applicatie zoals Unda effectief kan bijdragen aan het verbeteren van gezondheidstracking voor mensen met chronische aandoeningen. Het biedt een meerwaarde voor zowel gebruikers als zorgverleners, en vormt een veelbelovende basis voor verdere ontwikkeling binnen het domein van mHealth.


