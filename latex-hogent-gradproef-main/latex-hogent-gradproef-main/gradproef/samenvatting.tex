%%=============================================================================
%% Samenvatting
%%=============================================================================

% TODO: De "abstract" of samenvatting is een kernachtige (~ 1 blz. voor een
% thesis) synthese van het document.
%
% Een goede abstract biedt een kernachtig antwoord op volgende vragen:
%
% 1. Waarover gaat de graduaatsproef?
% 2. Waarom heb je er over geschreven?
% 3. Hoe heb je het onderzoek uitgevoerd?
% 4. Wat waren de resultaten? Wat blijkt uit je onderzoek?
% 5. Wat betekenen je resultaten? Wat is de relevantie voor het werkveld?
%
% Daarom bestaat een abstract uit volgende componenten:
%
% - inleiding + kaderen thema
% - probleemstelling
% - (centrale) onderzoeksvraag
% - onderzoeksdoelstelling
% - methodologie
% - resultaten (beperk tot de belangrijkste, relevant voor de onderzoeksvraag)
% - conclusies, aanbevelingen, beperkingen
%
% LET OP! Een samenvatting is GEEN voorwoord!

%%---------- Nederlandse samenvatting -----------------------------------------
%
% TODO: Als je je graduaatsproef in het Engels schrijft, moet je eerst een
% Nederlandse samenvatting invoegen. Haal daarvoor onderstaande code uit
% commentaar.
% Wie zijn/haar graduaatsproef in het Nederlands schrijft, kan dit negeren, de inhoud
% wordt niet in het document ingevoegd.

\IfLanguageName{english}{%
\selectlanguage{dutch}
\chapter*{Samenvatting}
\lipsum[1-4]
\selectlanguage{english}
}{}

%%---------- Samenvatting -----------------------------------------------------
% De samenvatting in de hoofdtaal van het document

\chapter*{\IfLanguageName{dutch}{Samenvatting}{Abstract}}

Chronische gezondheidsproblemen, zoals endometriose, migraine of prikkelbare darmsyndroom, vereisen vaak een nauwgezette opvolging van symptomen en levensstijl. Veel patiënten gebruiken momenteel verschillende applicaties of handmatige notities om gegevens over menstruatie, voeding, pijn en medicatie bij te houden. Deze versnippering leidt tot verlies aan overzicht, inconsistente registratie en moeilijkheden bij het delen van informatie met zorgverleners.

Deze graduaatsproef onderzocht hoe een mobiele applicatie kan bijdragen aan een efficiënte en overzichtelijke registratie van gezondheidsdata voor mensen met chronische aandoeningen. De centrale onderzoeksvraag luidde: \textbf{Hoe kan een mobiele applicatie bijdragen aan een efficiënte en overzichtelijke registratie van menstruatiecycli, voeding, pijnklachten en medicatieinname bij gebruikers met chronische gezondheidsproblemen?}

Het doel van dit project was het ontwikkelen van een gebruiksvriendelijk en functioneel prototype van een mobiele app die verschillende vormen van gezondheidstracking integreert. Hiervoor werd eerst een literatuuronderzoek uitgevoerd, gevolgd door een analyse van bestaande applicaties. Op basis daarvan werd een functionele analyse opgesteld en vervolgens een app ontwikkeld in Flutter, een cross-platform framework.

Het resultaat is de “Unda Health Tracker”, een werkend prototype waarin gebruikers op één centrale plek hun symptomen en levensstijl kunnen registreren. De applicatie biedt een visueel overzicht van ingevoerde data, ondersteunt het herkennen van patronen, en maakt het mogelijk gegevens te delen met zorgverleners. De keuze voor Flutter resulteerde in een technisch stabiele en visueel aantrekkelijke app die inzetbaar is op meerdere platformen.

De conclusie van het onderzoek is dat een geïntegreerde app zoals Unda een duidelijke meerwaarde kan bieden voor het dagelijks gezondheidsbeheer van mensen met chronische klachten. Niet alleen helpt het gebruikers bij het monitoren van hun gezondheid, het ondersteunt ook zorgverleners door objectieve en gestructureerde data te leveren. Toekomstige uitbreidingen kunnen zich richten op personalisatie, AI-analyse van trends en bredere klinische testen.


