%%=============================================================================
%% Inleiding
%%=============================================================================

\chapter{\IfLanguageName{dutch}{Inleiding}{Introduction}}%
\label{ch:inleiding}

Chronische gezondheidsproblemen vereisen vaak een actieve betrokkenheid van patiënten bij het opvolgen van hun eigen gezondheid. 
Artsen raden vaak aan om gegevens zoals menstruatiecycli, eet- en drinkgewoonten, pijnervaringen en medicatie-inname systematisch bij te houden. 
In de praktijk maken gebruikers hiervoor echter gebruik van meerdere apps of papieren notities, wat leidt tot een gefragmenteerd overzicht en verlies van informatie. 
Deze versnippering belemmert het herkennen van patronen en maakt het moeilijker om relevante gegevens te delen met zorgverleners. 
De Unda Health Tracker is ontstaan vanuit deze nood aan een centrale en gebruiksvriendelijke oplossing. De mobiele applicatie integreert meerdere vormen van gezondheidstracking in één platform. 
Het is gericht op personen met chronische aandoeningen of hormonale schommelingen, maar ook op gezondheidsbewuste gebruikers die hun levensstijl willen monitoren. 
Deze doelgroep heeft baat bij een duidelijke, visuele weergave van gezondheidsdata om meer inzicht te krijgen in hun klachten en evolutie. 
Het doel van deze graduaatsproef is dan ook het ontwikkelen van een werkend prototype van deze applicatie, met aandacht voor gebruiksgemak en overzichtelijkheid. 

\section{\IfLanguageName{dutch}{Probleemstelling}{Problem Statement}}%
\label{sec:probleemstelling}

Gebruikers die medische gegevens willen bijhouden, stuiten vaak op het probleem dat ze verschillende apps 
moeten combineren of terugvallen op handmatische notities. 
Deze kunnen verloren raken of tot verwarring leiden. 
Er is ook kans op verlies aan overzicht en inconsistentie in de dataregistratie. 
Daarnaast blijkt uit gesprekken met enkele artsen dat veel patiënten moeite hebben om correcte en volledige informatie aan te leveren tijdens consultaties. 
De huidige markt biedt voornamelijk apps die zich richten op een specifiek aspect, zoals menstruatie of voeding. 
Er is dus een meerwaarde in het ontwikkelen van een applicatie die deze functionaliteiten combineert 
en gebruiksvriendelijk maakt voor een afgebakende doelgroep: personen met chronische klachten of hormonale schommelingen die baat hebben bij 
een nauwkeurige en centrale registratie van hun gezondheidsdata. 

\section{\IfLanguageName{dutch}{Onderzoeksvraag}{Research question}}%
\label{sec:onderzoeksvraag}

De centrale onderzoeksvraag van deze graduaatsproef luidt:

\begin{quote}
    Hoe kan een mobiele applicatie bijdragen aan een efficiënte en overzichtelijke registratie van menstruatiecycli, voeding, pijnklachten en medicatie-inname bij gebruikers met chronische gezondheidsproblemen?
\end{quote}

Deelvragen hierbij zijn onder andere: 
\begin{itemize}
    \item Welke functies hebben gebruikers effectief nodig voor gezondheidstracking?
    \item Hoe kan de interface intuïtief en laadgrempelig worden ontworpen?
    \item Hoe kunnen de geregistreerde gegevens visueel worden weergegeven zodat ze patronen blootleggen?
\end{itemize}

\section{\IfLanguageName{dutch}{Onderzoeksdoelstelling}{Research objective}}%
\label{sec:onderzoeksdoelstelling}
De doelstelling van deze graduaatsproef is het ontwikkelen van een functioneel prototype van een mobiele applicatie die verschillende gezondheidstrackers integreert in één systeem. 
De applicatie moet gebruiksvriendelijk, visueel overzichtelijk en technisch stabiel zijn. Ze moet gebruikers in staat stellen om eenvoudig hun gegevens in te voeren, trends te herkennen en informatie te exporteren of delen met hun zorgverleners.

Een succesvolle uitvoering van dit project resulteert in: 
\begin{itemize}
    \item Een werkende Flutter-applicatie voor Android (en optioneel iOS),
    \item Een interface die verschillende vormen van gezondheidstracking combineert,
    \item Een eenvoudige, duidelijke visualisatie van de ingevoerde data,
    \item Een positieve evaluatie door minstens één arts of zorgverlener. 
\end{itemize}

\section{\IfLanguageName{dutch}{Opzet van deze graduaatsproef}{Structure of this associate thesis}}%
\label{sec:opzet-graduaatsproef}

% Het is gebruikelijk aan het einde van de inleiding een overzicht te
% geven van de opbouw van de rest van de tekst. Deze sectie bevat al een aanzet
% die je kan aanvullen/aanpassen in functie van je eigen tekst.

De rest van deze graduaatsproef is als volgt opgebouwd:

In Hoofdstuk~\ref{ch:stand-van-zaken} wordt op basis van literatuur en bestaande toepassigen de stand van zaken binnen het domein van gezondheidstracking besproken. 

In Hoofdstuk~\ref{ch:methodologie} wordt de methodologie toegelicht en worden de gebruikte onderzoekstechnieken besproken om een antwoord te kunnen formuleren op de onderzoeksvragen.

% TODO: Vul hier aan voor je eigen hoofstukken, één of twee zinnen per hoofdstuk

In Hoofdstuk~\ref{ch:conclusie}, tenslotte, wordt een conclusie geformuleerd en een antwoord gegeven op de onderzoeksvraag. Er wordt afgesloten met suggesties voor verder onderzoek en uitbreiding van de applicatie. 