%%=============================================================================
%% Methodologie
%%=============================================================================


\chapter{\IfLanguageName{dutch}{Methodologie}{Methodology}}%
\label{ch:methodologie}

In dit hoofdstuk wordt het plan van aanpak toegelicht dat gevolgd werd tijdens het uitvoeren van deze graduaatsproef. Het onderzoek werd opgedeeld in verschillende fasen, met telkens een duidelijk doel en bijhorende werkwijze. Elke fase bouwt voort op de vorige, met als einddoel het ontwikkelen van een werkend prototype in Flutter.

\section{Literatuuronderzoek}

De eerste stap bestond uit een grondige literatuurstudie. Hierbij werd informatie verzameld over Flutter als framework, de werking van de programmeertaal Dart, en de manier waarop Flutter apps compileert voor verschillende platformen. Naast officiële documentatie van Google werden ook tutorials, artikelen en video’s geraadpleegd om inzicht te krijgen in best practices en architecturale keuzes binnen Flutter-projecten.

\section{Analyse van bestaande applicaties}

Om inzicht te krijgen in goede ontwerpprincipes en gebruikerservaring, werden verschillende bestaande mobiele applicaties gedownload en geanalyseerd. Daarbij werd gelet op welke elementen gebruiksvriendelijk waren, welke navigatiestructuren werden toegepast, en welke onderdelen ik eventueel in mijn eigen applicatie wilde verwerken. Ook werd genoteerd welke elementen minder goed werkten of verwarrend waren voor de gebruiker. Deze analyse hielp bij het opstellen van een lijst met gewenste en ongewenste functionaliteiten.

\section{Functionele analyse en opzet}

Op basis van de inzichten uit het literatuuronderzoek en de applicatie-analyse werd een functionele analyse opgesteld. Deze bevatte een lijst van noodzakelijke onderdelen, zoals schermindelingen, inputmethodes, dataflows, en eventuele interacties met externe services of databronnen. Op basis van deze analyse werd een ruwe architectuur van de applicatie uitgetekend.

\section{Ontwikkelfase}

Na de voorbereidingsfase werd gestart met het ontwikkelen van de applicatie in Flutter. Tijdens het programmeerproces werd regelmatig teruggegrepen naar documentatie, tutorials en videomateriaal om bepaalde problemen of concepten beter te begrijpen. De ontwikkeling gebeurde iteratief: eerst werd een basisstructuur opgezet, waarna stap voor stap functionaliteiten en visuele elementen werden toegevoegd en verbeterd.

\section{Tussentijdse bijsturing}

Tijdens de implementatie werden voortdurend keuzes geëvalueerd en bijgestuurd waar nodig. Sommige ideeën bleken in de praktijk moeilijk implementeerbaar of minder gebruiksvriendelijk dan verwacht, en werden aangepast of vervangen. Dankzij deze flexibele werkwijze kon de applicatie stelselmatig worden verfijnd.

\section{Verantwoording}

De gekozen methodologie laat toe om zowel theoretisch als praktisch inzicht te verwerven in het onderwerp. De combinatie van literatuuronderzoek, analyse van bestaande oplossingen en hands-on ontwikkeling biedt een sterke basis om gefundeerde keuzes te maken tijdens het bouwproces van de applicatie.



