%%=============================================================================
%% Voorwoord
%%=============================================================================

\chapter*{\IfLanguageName{dutch}{Woord vooraf}{Preface}}%
\label{ch:voorwoord}

Deze graduaatsproef is voor mij meer dan een academisch project; het is geboren uit een persoonlijke zoektocht. Al geruime tijd kamp ik met onverklaarbare pijn waarvoor tot op vandaag nog geen sluitende diagnose werd gevonden. Tijdens deze weg langs verschillende artsen kreeg ik vaak het advies om een voedingsdagboek en pijnlogboek bij te houden om inzicht te krijgen in mogelijke triggers en verbanden, bijvoorbeeld met mijn menstruatiecyclus. Wat mij telkens opviel, was hoe omslachtig en versnipperd die registratie verliep. Apps voor voeding, andere voor pijn, en dan weer losse notities voor andere symptomen. In een periode waarin je je al kwetsbaar voelt, maakt dat het bijhouden van je gezondheid extra zwaar.

Met deze applicatie, die ik de naam \textit{Unda} heb gegeven, Latijn voor 'golf', wil ik iets teruggeven. Gezondheid is zelden een rechte lijn; ze beweegt in golven van goede en minder goede momenten. Voor mensen met chronische klachten zijn die golven vaak overweldigend. \textit{Unda} wil een hulpmiddel zijn om rust te brengen in de chaos, door alles op één plek overzichtelijk te registreren en te delen met zorgverleners.

Het kleurenthema van de applicatie is groen, geïnspireerd door de Chinese betekenis van de kleur: hoop. Want hoop is vaak het eerste wat mensen met chronische pijn dreigen te verliezen. Met deze app wil ik een klein beetje hoop aanreiken, in de vorm van houvast, inzicht en erkenning.

Ik wil uit de grond van mijn hart mijn partner, familie en vrienden bedanken voor hun eindeloze steun, liefde en aanmoediging. Ook mijn arts wil ik bedanken voor het meezoeken naar antwoorden, en het motiveren om mijn klachten serieus te nemen. Tot slot een bijzonder dankwoord aan mijn leerkrachten voor hun begeleiding, hun kennis en hun vertrouwen in mijn kunnen. Zonder hen had dit project nooit tot leven kunnen komen.

